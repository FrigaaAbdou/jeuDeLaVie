\documentclass[12pt,a4paper]{article}
\usepackage[utf8]{inputenc}
\usepackage[T1]{fontenc}
\usepackage{geometry}
\geometry{margin=2.5cm}
\usepackage{enumitem}
\usepackage{hyperref}
\usepackage{courier}
\usepackage{listings}
\lstset{
  basicstyle=\ttfamily\small,
  columns=fullflexible,
  keepspaces=true,
  showstringspaces=false
}

\title{Jeu de la vie --- Tests unitaires}
\author{Projet C++17 (Grid / GameOfLife / ConwayRule)}
\date{}

\begin{document}
\maketitle

\section{Objectif}
Documenter les tests unitaires ajoutés pour la logique du Jeu de la vie (sans SFML). Les tests valident la règle de Conway, la stabilité et quelques cas limites courants.

\section{Architecture du runner}
\begin{itemize}[leftmargin=1.1em]
  \item Fichier \texttt{tests/main.cpp} : harness minimaliste (pas de framework externe).
  \item Compilation via \texttt{make test} qui produit \texttt{tests/test\_runner} en liant uniquement les fichiers de logique (\texttt{Grid.cpp}, \texttt{GameOfLife.cpp}, \texttt{InitialStateLoader.cpp}, \texttt{GridExporter.cpp}).
  \item Aucun lien avec SFML : exécution en ligne de commande uniquement.
  \item Arrêt immédiat sur échec (\texttt{std::exit(1)}) pour remonter clairement la première assertion en défaut.
\end{itemize}

\section{Harness et helpers}
\begin{itemize}[leftmargin=1.1em]
  \item \texttt{expect(cond, message)} : vérifie une condition, affiche ``Test failed: \ldots'' et quitte en cas d'échec.
  \item \texttt{logCase}/\texttt{logOk} : traces lisibles en console pour chaque cas de test.
  \item \texttt{makeGrid(vector<string>)} : construit une grille en mémoire à partir de lignes ``0/1'' (``1'' = cellule vivante).
  \item \texttt{expectGrid(grid, rows, msg)} : compare une grille avec un motif attendu (dimensions et états des cellules).
\end{itemize}

\section{Cas de test couverts}
\begin{enumerate}[leftmargin=1.1em]
  \item \textbf{Bloc (still life) stable} \\
    Motif $2\times2$ vivant entouré de cellules mortes. Après un pas de simulation :
    \begin{itemize}
      \item Le motif reste identique (vérification par \texttt{expectGrid}).
      \item \texttt{isStable()} retourne vrai (grille inchangée).
      \item \texttt{hasFinished()} retourne vrai (stabilité atteinte).
    \end{itemize}
  \item \textbf{Blinker (oscillateur période 2)} \\
    Ligne de trois cellules vivantes. Après un pas, elle devient colonne; après deux pas, revient à l'horizontale. Les deux phases sont comparées à des motifs attendus.
  \item \textbf{Cellule isolée qui meurt} \\
    Une cellule vivante sans voisins doit disparaître au pas suivant (mort par sous-population).
\end{enumerate}

\section{Extrait de code (harness)}
\begin{lstlisting}[language=C++]
static void expect(bool condition, const std::string& message) {
    if (!condition) {
        std::cerr << "Test failed: " << message << "\n";
        std::exit(1);
    }
}

static Grid makeGrid(const std::vector<std::string>& rows) {
    Grid g(rows.size(), rows.front().size(), false);
    for (int y = 0; y < g.rows(); ++y)
        for (int x = 0; x < g.cols(); ++x)
            g.at(y, x).setState(rows[y][x] == '1'
                ? static_cast<CellState*>(new AliveState())
                : static_cast<CellState*>(new DeadState()));
    return g;
}
\end{lstlisting}

\section{Exécution}
\begin{itemize}[leftmargin=1.1em]
  \item \texttt{make test} pour compiler le binaire \texttt{tests/test\_runner}.
  \item \texttt{tests/test\_runner} pour exécuter les cas. Sortie attendue :
\end{itemize}
\begin{lstlisting}
Running Game of Life unit tests...
[CASE] Still life (block) remains stable... OK
[CASE] Blinker oscillates with period 2... OK
[CASE] Isolated live cell dies in one step... OK
All tests passed.
\end{lstlisting}

\section{Extensions possibles}
\begin{itemize}[leftmargin=1.1em]
  \item Ajouter un test de mode torique (\texttt{Grid} avec \texttt{toroidal=true}) pour vérifier le comptage des voisins.
  \item Ajouter des tests d'échec du chargeur de fichier (\texttt{InitialStateLoader}) pour détecter un format invalide.
  \item Couvrir des motifs plus longs (planeur, pulsar) et la détection de stabilité après plusieurs pas.
  \item Intégrer un framework léger (Catch2/doctest) si les besoins de reporting/fixtures augmentent.
\end{itemize}

\end{document}
